%%%%%%%% ICML 2021 EXAMPLE LATEX SUBMISSION FILE %%%%%%%%%%%%%%%%%

\documentclass{article}

% Recommended, but optional, packages for figures and better typesetting:
\usepackage{microtype}
\usepackage{graphicx}
\usepackage{subfigure}
\usepackage{booktabs} % for professional tables

% hyperref makes hyperlinks in the resulting PDF.
% If your build breaks (sometimes temporarily if a hyperlink spans a page)
% please comment out the following usepackage line and replace
% \usepackage{icml2021} with \usepackage[nohyperref]{icml2021} above.
\usepackage{hyperref}

% Attempt to make hyperref and algorithmic work together better:
\newcommand{\theHalgorithm}{\arabic{algorithm}}

% Use the following line for the initial blind version submitted for review:
\usepackage[accepted]{icml2021}
\usepackage{natbib}


% If accepted, instead use the following line for the camera-ready submission:
%\usepackage[accepted]{icml2021}

% The \icmltitle you define below is probably too long as a header.
% Therefore, a short form for the running title is supplied here:
\icmltitlerunning{}

\begin{document}

\twocolumn[
\icmltitle{ Music Controllable Diffusion\\
           Generating MIDI Music using Controllable Diffusion }

% It is OKAY to include author information, even for blind
% submissions: the style file will automatically remove it for you
% unless you've provided the [accepted] option to the icml2021s
% package.

% List of affiliations: The first argument should be a (short)
% identifier you will use later to specify author affiliations
% Academic affiliations should list Department, University, City, Region, Country
% Industry affiliationqs should list Company, City, Region, Country

% You can specify symbols, otherwise they are numbered in order.
% Ideally, you should not use this facility. Affiliations will be numbered
% in order of appearance and this is the preferred way.
\icmlsetsymbol{equal}{*}

\begin{center}
\begin{tabular}{rl}
Name: & Saravana Kumar Rathinam \\
SUNet ID: & 06512865 \\
\end{tabular}
\end{center}


\icmlcorrespondingauthor{Saravana Rathinam}{saravanr@stanford.edu}

% You may provide any keywords that you
% find helpful for describing your paper; these are used to populate
% the "keywords" metadata in the PDF but will not be shown in the document
\icmlkeywords{Machine Learning, ICML}

\vskip 0.3in
]

% this must go after the closing bracket ] following \twocolumn[ ...

% This command actually creates the footnote in the first column
% listing the affiliations and the copyright notice.
% The command takes one argument, which is text to display at the start of the footnote.
% The \icmlEqualContribution command is standard text for equal contribution.
% Remove it (just {}) if you do not need this facility.

%\printAffiliationsAndNotice{}  % leave blank if no need to mention equal contribution
%\printAffiliationsAndNotice{\icmlEqualContribution} % otherwise use the standard text.

\begin{abstract}
Diffusion based generative models can be used to generate music in a controllable way. Generating long music sequences from raw audio waveforms can be very compute intensive. MIDI format gives a much more compressed representation of a subset of Music which may give us a tractable way to generate melodies.
\end{abstract}

\section{Motivation}
\label{submission}
Composing music is a skill that is acquired by many years of practice. The music itself is the result of the life experiences of the musician, their state of mind, their unconcious and concious thoughts. Their creative talent is subjective and difficult to generalize. Recent advances in Generative models for Music generation have shown impressive results where the focus has been to replace the creative process. Learning the distribution of music creation may be an intractable problem at the moment. However one approach we can take is to build tools that serve as an aid in the creative process.  If a musician already has a few ideas in mind on how a song or melody should start, can the problem be modelled as a conditional generative process where given the start and style of the song, can a model generate multiple possibilities of how the song can proceed? 

In such a generative system, the inputs to the model would be a short MIDI sequence. The system would generate a bunch of sequences that may serve as suggested next sequences and so on. By conditioning on the input and letting the musician choose the path to take, the model can help in the creative process.

Such a tool can help naive music enthusiasts to try creating music. Consider the spectrum of music creation tools, on one end are very sophisticated tools like Abelton Live, FL Studio which are used by trained musicians. Then there are tools in the middle like Garage band for casual users. With conditional music generation we can build a tool that lies at the other end of the spectrum, enabling anybody to try music creation.

\section{Related Works}

There has been a lot of good work in the area of Music creation. Google's Magenta project explores the role of machine learning in the creative process. In a recent papers \citep{mittal} built a multi-stage non autoregressive generative model that enabled using diffusion models on discrete data. They generated both unconditional music as well as conditional in-filling. They used a Denoising Diffusion Probabilistic Model \cite{ho2020denoising} on top of a MusicVAE model that generated the continuous time latent embeddings. Similarly \cite{choi2021ilvr} proposed an Iterative Latent Variable Refinement (ILVR) method to guide the DDPM to generate high quality images based on a given reference image. Also \cite{song2020denoising}  produced a way to accelerate sampling process of a DDPM which can make generation process of sequences faster. In another beautiful apprach, \cite{bazin2021} built an interactive web interface that transforms sound by inpainting. This approach is similar to what \cite{meng2021sdedit} built with SDEdit that adapts to editing tasks at test time, without the need for re-training the model.

\begin{itemize}
\item Submissions must be in PDF\@.
\item Submitted papers can be up to eight pages long, not including references, plus unlimited space for references. Accepted papers can be up to nine pages long, not including references, to allow authors to address reviewer comments. Any paper exceeding this length will automatically be rejected. 
\item \textbf{Do not include author information or acknowledgements} in your
    initial submission.
\item Your paper should be in \textbf{10 point Times font}.
\item Make sure your PDF file only uses Type-1 fonts.
\item Place figure captions \emph{under} the figure (and omit titles from inside
    the graphic file itself). Place table captions \emph{over} the table.
\item References must include page numbers whenever possible and be as complete
    as possible. Place multiple citations in chronological order.
\item Do not alter the style template; in particular, do not compress the paper
    format by reducing the vertical spaces.
\item Keep your abstract brief and self-contained, one paragraph and roughly
    4--6 sentences. Gross violations will require correction at the
    camera-ready phase. The title should have content words capitalized.
\end{itemize}



% In the unusual situation where you want a paper to appear in the
% references without citing it in the main text, use \nocite


\bibliography{project}
\bibliographystyle{icml2021}


%%%%%%%%%%%%%%%%%%%%%%%%%%%%%%%%%%%%%%%%%%%%%%%%%%%%%%%%%%%%%%%%%%%%%%%%%%%%%%%
%%%%%%%%%%%%%%%%%%%%%%%%%%%%%%%%%%%%%%%%%%%%%%%%%%%%%%%%%%%%%%%%%%%%%%%%%%%%%%%
% DELETE THIS PART. DO NOT PLACE CONTENT AFTER THE REFERENCES!
%%%%%%%%%%%%%%%%%%%%%%%%%%%%%%%%%%%%%%%%%%%%%%%%%%%%%%%%%%%%%%%%%%%%%%%%%%%%%%%
%%%%%%%%%%%%%%%%%%%%%%%%%%%%%%%%%%%%%%%%%%%%%%%%%%%%%%%%%%%%%%%%%%%%%%%%%%%%%%%
\appendix
\section{Do \emph{not} have an appendix here}

\textbf{\emph{Do not put content after the references.}}
%
Put anything that you might normally include after the references in a separate
supplementary file.

We recommend that you build supplementary material in a separate document.
If you must create one PDF and cut it up, please be careful to use a tool that
doesn't alter the margins, and that doesn't aggressively rewrite the PDF file.
pdftk usually works fine. 

\textbf{Please do not use Apple's preview to cut off supplementary material.} In
previous years it has altered margins, and created headaches at the camera-ready
stage. 
%%%%%%%%%%%%%%%%%%%%%%%%%%%%%%%%%%%%%%%%%%%%%%%%%%%%%%%%%%%%%%%%%%%%%%%%%%%%%%%
%%%%%%%%%%%%%%%%%%%%%%%%%%%%%%%%%%%%%%%%%%%%%%%%%%%%%%%%%%%%%%%%%%%%%%%%%%%%%%%


\end{document}


% This document was modified from the file originally made available by
% Pat Langley and Andrea Danyluk for ICML-2K. This version was created
% by Iain Murray in 2018, and modified by Alexandre Bouchard in
% 2019 and 2021. Previous contributors include Dan Roy, Lise Getoor and Tobias
% Scheffer, which was slightly modified from the 2010 version by
% Thorsten Joachims & Johannes Fuernkranz, slightly modified from the
% 2009 version by Kiri Wagstaff and Sam Roweis's 2008 version, which is
% slightly modified from Prasad Tadepalli's 2007 version which is a
% lightly changed version of the previous year's version by Andrew
% Moore, which was in turn edited from those of Kristian Kersting and
% Codrina Lauth. Alex Smola contributed to the algorithmic style files.
